% LaTeX Syllabus Template


% Copyright (C) 2004-2018 Ricardo Ceneviva 
% http://ricardoceneviva.com


% You may use use this document as a template to create your own Syllabus
% and you may redistribute the source code freely. No attribution is
% required in any resulting documents. I do ask that you please leave
% this notice and the above URL in the source code if you choose to
% redistribute this file.


\documentclass[a4paper, 12pt]{article}


% Comment the following lines to use the default Computer Modern font
% instead of the Palatino font provided by the mathpazo package.
% Remove the 'osf' bit if you don't like the old style figures.

\usepackage[sc,osf]{mathpazo}

\usepackage[utf8x]{inputenc}
\usepackage{ucs}  % allow for non-latin characters

%LANGUAGE == Port(br)

\usepackage[brazil]{babel}     %esse pacote é para hifenização
\usepackage[T1]{fontenc}

\usepackage{bold-extra}            %Activate to use  bold small capitals
%\setlength{\textheight}{9.5in}      % increase text height to fit on 1-page 
\usepackage{ae,aeguill}             %to obtain a good looking PDF document:

\usepackage[parfill]{parskip}    % Activate to begin paragraphs with an empty line rather than an indent

%\usepackage{fancyhdr}        % adds header & footer

% Set your name here
\def\name{Ricardo Ceneviva}

% Replace this with a link to your Course website if you like, or set it empty
% (as in \def\footerlink{}) to remove the link in the footer:
%\lhead{Ricardo Ceneviva}
%\rhead{Métodos}
%\rfoot{\thepage}
\def\footerlink{http://ricardoceneviva.com/Matematica/}

\usepackage{hyperref}             % The following metadata will show up in the PDF properties
\hypersetup{
  colorlinks = true,
  urlcolor = black,
  pdfauthor = {\name},
  pdfkeywords = {Matematica, nivel básico, iesp-uerj},
  pdftitle = {\name: Syllabus},
  pdfsubject = {Political Methodology},
  pdfpagemode = UseNone
}

\usepackage{geometry}
\geometry{
  body={6.5in, 9.0in},
  left=1.0in,
  top=1.25in
}

% Customize page headers
\pagestyle{myheadings}
\markright{\name}
\thispagestyle{empty}

% Custom section fonts
\usepackage{sectsty}
\sectionfont{\rmfamily\mdseries\Large\bf}
\subsectionfont{\rmfamily\mdseries\itshape\large}

% Other possible font commands include:
% \ttfamily for teletype,
% \sffamily for sans serif,
% \bfseries for bold,
% \scshape for small caps,
% \normalsize, \large, \Large, \LARGE sizes.

% Don't indent paragraphs.
\setlength\parindent{0em}

% Make lists without bullets
\renewenvironment{itemize}{
  \begin{list}{}{
    \setlength{\leftmargin}{1.5em}
  }
}{
  \end{list}
}

\begin{document}
\pagestyle{plain}

% Place name at left
%{ \center \huge \name}

% Alternatively, print name centered and bold:
%\centerline{\huge \bf \name}

\begin{center}
\begin{bf} 

\href{http://www.iesp.uerj.br/}{Instituto de Estudos Sociais e Políticos} \\

Universidade do Estado do Rio de Janeiro \\


\vspace{0.5cm}

\Large

\Large{Matemática Básica para Ciências Socais}

\vspace{0.5cm}

\normalsize


1º Semestre de 2017 \\
\vspace{0.5cm}


Professor: Ricardo Ceneviva \\
\href{mailto:ceneviva@iesp.uerj.br}{\tt ceneviva@iesp.uerj.br} \\
\vspace{0.25cm}

Monitora: Cynthia Cunha \\
\href{mailto:cunha.cynthia@gmail.com}{\tt cunha.cynthia@gmail.comr} \\
\vspace{0.5cm}

Carga Horária: 15 horas \\
 
\end{bf}
\end{center}


\section*{Objetivos do Curso}

O objetivo deste curso é promover e uma revisão de conceitos básicos de matemática e um nivelamento entre os estudantes que cursarão ``Introdução à Análise de Dados'' (Lego I). Este curso é destinado aos alunos que necessitam de uma revisão intensiva em conceitos matemáticos básicos, que são essenciais para um bom aproveitamento do curso de Lego I e, mais amplamente, para uma melhor compreensão de análise quantitativa nas pesquisas em Ciências Sociais, com um especial foco em Sociologia, Ciência Política e Relações Internacionais.

Este mini-curso de ``Matemática Básica para Ciências Sociais'' será realizado na primeira semana (de 07 a 10 de março) do curso regular de Introdução à Análise de Dados, com sessões pelas manhãs, das 9:00h às 12:00h, e funcionará como um curso preparatório para Lego I. Ele cobrirá temas como a notação matemática, teoria dos conjuntos,  sistemas numéricos, álgebra, funções, e soluções para sistemas de equações lineares. Este mini-curso de nivelamento é obrigatório para TODOS os alunos que pretendam cursar Introdução a Análise de Dados.

\section*{Pré-requisitos}

Não há pré-requisitos para esse curso.  


\section*{Sumário do curso}

\begin{enumerate}

\item Preliminares e Revisão de Álgebra 
\item Funções, Limites e Utilidade
\item Derivadas 
\item Integrais
\item Teoria dos Conjuntos e Probabilidades

\end{enumerate}


\vspace{1.0cm}

\section*{Programa}

\subsection*{\textbf{Aula 1} (dia 07/03) : ``Notação Matemática e Revisão de Algebra''} 
\begin{itemize}

\item Gill, Jeff. (2006). Essential Mathematics for Political and Social Research.  \textbf{chapter 1, sections 1.1, 1.2, 1.3 and 1.4.}


\item Moore, Will, and David Siegel. (2013). A Mathematics Course for Political and Social Research. \textbf{chapter 1.}

\end{itemize}


\subsection*{\textbf{Aula 2} (dia 08/03): ``Funções, Equações e Utilidade''} 
\begin{itemize}

\item \textbf{Leitura Básica}

\item Moore, Will, and David Siegel. 2013. A Mathematics Course for Political and Social Research, \textbf{capítulo 3}

\item \textbf{Leitura Complementar}

\item Gill, Jeff. (2006). Essential Mathematics for Political and Social Research, \textbf{capítulo 1, seções 1.5, 1.6, 1.7, 1.8 e 1.9.}

\item Lista de exercícios 1 

\end{itemize}



\subsection*{\textbf{Aula 3} (dia 09/03): ``Limites e Derivadas''} 
\begin{itemize}

\item \textbf{Leitura Básica}

\item Moore, Will, and David Siegel. 2013. A Mathematics Course for Political and Social Research, \textbf{capítulos 4, 5 e 6.}

\item \textbf{Leitura Complementar}

\item Gill, Jeff. (2006). Essential Mathematics for Political and Social Research,  \textbf{capítulo 5.}

\item Lista de exercícios 2

\end{itemize}


\subsection*{\textbf{Aula 4} (dia 10/03): ``Integrais e Outras Regras de Derivação''} 
\begin{itemize}

\item \textbf{Leitura Básica}

\item Moore, Will, and David Siegel. 2013. A Mathematics Course for Political and Social Research, \textbf{capítulos 7 e 8, seção 8.2.}

\item \textbf{Leitura Complementar}

\item Gill, Jeff. (2006). Essential Mathematics for Political and Social Research,  \textbf{capítulo 6.}

\item Lista de exercícios 3

\end{itemize}



%\subsection*{\textbf{Aula 5} (dia 11/03): ``Teoria dos Conjuntos e Noções de Probabilidade''} 
%\begin{itemize}

%\item \textbf{Leitura Básica}

%\item Moore, Will, and David Siegel. 2013. A Mathematics Course for Political and Social Research, \textbf{capítulos 8, 15 e 16.}

%\item \textbf{Leitura Complementar}

%\item Gill, Jeff. (2006). Essential Mathematics for Political and Social Research,  \textbf{capítulo 7.}

%\item Lista de exercícios 4

\%end{itemize}


\vspace{1.0cm}

\section*{Referências Bibliográficas}
%\vspace{1.0cm}
\begin{enumerate}

\item \textsc{Gill}, Jeff. (2006). \emph{Essential Mathematics for Political and Social Research}. 1st Edition, Cambridge University Press.

\vspace{0.5cm}

\item \textsc{Moore}, Will, and David \textsc{Siegel}. (2013). \emph{A Mathematics Course for Political and Social Research}. 1st edition, Princeton, N.J.: Princeton University Press.

\vspace{0.5cm}

\item \textsc{Simon}, Carl, and Lawrence \textsc{Blume}. (2004). \emph{Matemática para Economistas.} Porto Alegre, Editora Bookman. 

\end{enumerate}



%\bigskip
 
\vspace{1cm}

% Footer
\begin{center}
  \begin{footnotesize}
    Versão preliminar e sujeita a  alterações, atualizada em: \today. \\
    Documento produzido em \LaTeX  \\
    
    \href{\footerlink}{\texttt{\footerlink}}
  \end{footnotesize}
\end{center}

\end{document}